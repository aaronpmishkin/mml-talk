%! TEX root = ../../main.tex

%% Illustration of cone decomposition. 

\begin{tikzpicture}[scale=1,
		declare function={
				cone_1(\x)= 10*\x;
				cone_2(\x)= -\x;
				cone_3(\x)= -4*\x;
				bounds(\x)= \x - 10;
			}
	]
	\begin{axis}[width=\linewidth, height=6cm,
			axis lines=center, yticklabels={,,}, xticklabels={,,},
			ymin=-2, ymax=6, ytick={-2,...,5}, ylabel=$$, x axis line style={-},
				xmin=-6, xmax=6, xtick={-5,...,5}, xlabel=$$, y axis line style={-},
		]

		\draw [name path=cone_1, solid, line width=1pt] (axis cs:0,-2) -- (axis cs:0,6);
		\draw [name path=bounds, line width=0pt] (axis cs:-6,-2) -- (axis cs:-6,6);
		\tikzfillbetween[of=cone_1 and bounds, on layer=ft]{good, opacity=0.1};

		%% point labels
		% origin point
		\node[circle, fill, inner sep=1pt] at (axis cs:0,0) {};

		% active examples 
		\node[label=right:$(x_1\!\mathbin{,} y_1)$, circle, fill, inner sep=0.5mm] at (axis cs:-5,1) {};
		\node[label=right:$(x_2\!\mathbin{,} y_2)$, circle, draw, line width=0.25mm, inner sep=0.5mm] at (axis cs:0.5,5) {};


		% lines
		\draw [->, draw=bad, line width = 1mm] (axis cs:0,0) -- (axis cs:-1,0) node[near end,above] {$w^*_{1}$};

		\node[] at (axis cs:-4,5) { $ \lambda \uparrow \uparrow$ };
    \node[text width=13em, text centered] at (axis cs:3,2) { \bad{Solution path} \bad{is discontinuous!} };
	\end{axis}

\end{tikzpicture}%
