%! TEX root = ../../main.tex

%% Illustration of cone decomposition. 

\begin{tikzpicture}[scale=1,
		declare function={
				cone_1(\x)= \x;
				cone_2(\x)= -\x;
				cone_3(\x)= -4*\x;
			}
	]
	\begin{axis}[width=\linewidth, height=8cm,
			axis lines=center, yticklabels={,,}, xticklabels={,,},
			ymin=-4, ymax=4, ytick={-5,...,5}, ylabel=$$, x axis line style={-},
				xmin=-6, xmax=6, xtick={-5,...,5}, xlabel=$$, y axis line style={-},
		]
		\addplot[name path=cone_1, domain=-6:6, samples=100, line width=1pt]{cone_1(x)};
		\addplot[name path=cone_2, domain=-6:6, samples=200, line width=1pt]{cone_2(x)};
		%\addplot[name path=cone_3, domain=-6:6, samples=200, line width=1pt]{cone_3(x)};

		% add color fill to both cones.
		\addplot fill between[
				of = cone_1 and cone_2,
				split, % calculate segments
				every even segment/.style = {fill=blue, fill opacity=0},
				every odd segment/.style  = {fill=teal, fill opacity=0.3}
			];

		%% point labels
		% origin point
		\node[circle, fill, inner sep=1pt] at (axis cs:0,0) {};

		% labels
		\node[label={0:$\calK_j$}] at (axis cs:4,-1) {};

		% lines
        \draw [->, draw=red, line width = 0.4mm] (axis cs:2,2) -- (axis cs:3,1) node[midway,above right] {$S_{11} \cdot x_1$};
		\draw [->, draw=red, line width = 0.4mm] (axis cs:2,-2) -- (axis cs:3,-1) node[midway,below right] {$S_{22} \cdot x_2$};
		%\draw [->, draw=red, line width = 0.4mm] (axis cs:0.5,-2) -- (axis cs:-2.5,-2.75) node[pos=0.9,below right] {$S_{33} \cdot x_3$};
	\end{axis}

\end{tikzpicture}%
